% Report Template
%Compile with XeLaTeX
\documentclass[a4paper,11pt]{article}
\usepackage{fontspec}
\setmainfont{Arial}
%
\usepackage{amsmath}   %Better maths and more symbols
%
\usepackage[margin=1.5cm]{geometry}  %set margin width all around page
\usepackage{setspace}
\usepackage[compact]{titlesec}
%
\usepackage{graphics}
\usepackage{graphicx}
\graphicspath{/}  %Graphics saved in same directory as .tex file
\usepackage{float}  %Fancy graphics placement [H] [H!] arguments
%
\usepackage{caption}
%
\usepackage{multirow}  % Table cells spanning multiple rows
%
\usepackage{enumitem}    % Enumerated lists in the same format as bullet lists
%
\usepackage{natbib}    %Bibliography management  %Use author/date citations
\bibliographystyle{agsm}
\usepackage{url}
\usepackage{cite}
%
\usepackage{color}
\newcommand{\todo}[1]{\textcolor{red}{#1}}   % \todo{NOTE TO SELF WRITTEN IN RED}
%
\usepackage{lineno}
\linenumbers

%--------------------------------------------------------------
\begin{document}
%--------------------------------------------------------------
\setlength{\parskip}{10pt}   %space after paragraphs
\setlength{\headsep}{30pt} %space at page top
\setlength{\parindent}{0pt} %length of indent at start of par
%
\titlespacing{\section}{0pt}{*3}{*-0.1} %Position section titles  {indent}{*spaceabove}{*spacebelow}
\titlespacing{\subsection}{0pt}{*-0.5}{*0} %Position subsection titles  {indent}{*spaceabove}{*spacebelow}
\titlespacing{\subsubsection}{0pt}{*0}{*0} %Position subsubsection titles  {indent}{*spaceabove}{*spacebelow}
%
\pagenumbering{arabic}  %Add page numbering
%
\raggedright %align text to the left with long words taken to the next line
%--------------------------------------------------------------
\begin{center}{\LARGE{Changes in forest structure along an elevational gradient in the Peruvian Andes cause species-specific stress responses in tree seedlings}}\end{center}  %Insert Title

\section*{Abstract}
4 bullet points (1) research conducted + rationale, (2) central methods, (3) key results, (4) main conclusions including key points of discussion. 

\section*{Introduction}
Rapid anthropogenic climate change is causing many species, across a wide range of taxa, to shift their distributions in space \citep{Hughes2000, Parmesan2006, Chen2011}. \citet{Chen2011} estimates average latitudinal and poleward migration rates of \todo{NUM} and \todo{NUM}, respectively. For sessile taxa such as trees, range shifts occur as a result of differential recruitment and mortality over space \todo{REF}. Previous studies have suggested that the ability of tree species to respond to changes in mean annual temperature and precipitation regime will be important in determining species success over the coming century \citep{Colwell2008, Chen2011, Feeley2012}. Responses may occur either in the form of adaptation, i.e. changes in phenology, physiology and morphology, or through range shifts over space as species track changing temperatures (Bellard et al. 2012). Range shifts have been observed in many studies across the world \todo{REF}, indicating that climate change may be occurring too fast to allow evolutionary responses in the form of adaptation \todo{REF}. Predicting range shifts and more importantly the ability for species to shift their ranges across space has become an active field of research \todo{REFS FROM BELLARD 2012}. Understanding species range shifts can help conservation scientists to identfy species and species assemblages at risk of extinction and can inform strategies to mitigate the effects of climate change on biodiversity. 

The majority of efforts to predict species range shifts have used bioclimatic envelope models \citep{Pearson2003}. Bioclimatic envelopes are constructed by correlating current species ranges with observed environmental conditions within those boundaries, then projecting spatially explicit climate trends into the future under different climate change scenarios to predict how species boundaries will adjust in response (e.g. \citealt{Berry2002, Peterson2002, Thuiller2005, Araújo2006}) \citep{Sinclair2010}. When predicting range shifts and species success under a changing climate, it is important to consider that range shifts driven by a single environmental variable like mean annual temperature may cause recruitment of individuals into habitats that are suboptimal in other senses, if range shifts outstrip acclimatory/adaptive potential \todo{REF}. In these suboptimal habitats, range shifts may lead to reductions in local and regional species richness (Colwell et al. 2008), changes in community composition \todo{REF} ecosystem functioning (Bellard et al. 2012), and ecosystem service provision that are not predicted by the bioclimatic envelope models (Dobson et al. 2011, Isbell et al. 2011). Bioclimatic envelope models have been criticised for making a number of gross simplifications \todo{REF}, \todo{INCLUDE EXAMPLES?}. Basic bioclimatic envelope models assume that the breadth of the realised niche (observed spatial range) equals that of the fundamental niche (bioclimatic envelope) (Jump \& Pe\~{n}uelas 2005, Hoffmann \& Sgr\`{o} 2011). This assumption is challenged by studies which demonstrate that the realised niche is often smaller than the fundamental niche, owing to the presence of unmeasured environmental pressures, such as biotic interactions with other species (Davis et al. 1998, Van der Putten et al. 2010, Ettinger et al. 2011). In order to accurately predict range shifts and their consequences for future biodiversity, it is important to expand bioclimatic envelope models to include variables which describe habitat as well as climatic variables such as temperature and precipitation. 

For forest trees, particularly in moist tropical forests, there is often high levels of mortality during the seedling recruitment stage \todo{Coomes and Grubb 2000}. Many seedlings perish due tosuboptimal shade regime by adult trees todo{REF}. The seedlings of many tropical tree species are highly adapted to shade \todo{REF}. If a seedling germinates in an open space This mortality bottleneck provides a limiting factor to the success of tropical forest tree species experiencing range shifts. If seedlings germinate in areas that are only sparsely shaded, UV damage may occur leading to loss of photosynthetic capacity, reducing growth rates and occasionally resulting in seedling mortality \todo{REF}.  

In montane cloud forests, range shifts are occurring faster than in other areas \todo{WHY?}. As the temperature rises, plant species are being figuratively pushed up slope, with higher recruitment at the upslope edge of their range and higher mortality at the downslope edge of their range \todo{REF}. Particularly in the tropics, as altitude increases, UV-B concentration increases, with plants at high altitudes having specific adaptations to avoid UV-B damage to photosynthetic machinery. Species found at low altitudes however are less adapted to high UV-B environments, particularly at the seedling growth stage \todo{REF}. It therefore follows that as species move upslope in response to  

Tree species have overlapping generations, so the \todo{size composition and structure} of the adult population will impact the next generation of seedlings. Similarly, as species' ranges shift in response to climate change, adult trees from the previous generation will still be present, therefore affecting the recruitment and growth of the seedlings.

In this study, we investigated the effects of variation in adult tree structure on seedling growth form and photosynthetic stress, across an elevational gradient in the Peruvian Andes. We tested two hypotheses: 1) Seedlings growing at the upslope edge of their range would experience higher levels of photosynthetic stress than those at the downslope edges of their range, 2) Species would differ in their acclimatory responses to changes in forest structure across the elevational gradient. 

\section*{Materials and Methods}
\subsection*{Study Site}
Data collection was conducted across 10 permanent 1 ha closed canopy forest plots in the Kos\~{n}ipata Valley of Man\'{u} National Park, Peru (-13\textdegree N, -71\textdegree W, Figure \ref{fig:sites}, Table \ref{table:sitechar}). The Kos\~{n}ipata Valley has been identified as a potential migration corridor for lowland species to migrate to higher elevations in response to temperature increase \citep{Feeley2011} and so is an appropriate location to study range shift drivers. Plots are situated between 400 and 3200 m.a.s.l. along this migration corridor (Table \ref{table:sitechar}, Figure \ref{fig:range_plot}). The plots form part of a larger plot network established by the Andes Biodiversity and Ecosystem Research Group (ABERG) in 2003 \citep{Malhi2010, Girardin2013a}.

\subsection*{Study species} 
We chose nine tree species for comparison from a range of 1635 identified species within the 10 study plots. Species were selected according to their contrasting ranges (Figure \ref{fig:range_plot}), differences in genus migratory pattern \citep{Feeley2011}, and because each species makes up a dominant proportion of the biomass in plots across their ranges \todo{(ABERG, unpublished data, Appendix VI)}. Despite having no quantitative range shift prediction information, Iriartea deltoidea and Dictyocaryum lamarckianum were included in order to observe potential differences between monocot and dicot species, as both are monocots. Both \textit{I. deltoidea} and \textit{D. lamarckianum} are large-seeded palm species, as such, they are expected to be migrating upslope, similar to other large-seeded palms \citep{Hillyer2010}. Seedlings of \textit{Myrcia spp.} are difficult to reliably identify to species in the field and were thus sampled as a composite of three potential species: \textit{Myrcia splendens}, \textit{M. fallax}, and \textit{M. rostrata}, referred to as \textit{Myrcia spp.}. 

\subsection*{Sampling and Measurement}
Species were sampled in three plots representing the upper, middle and lower elevational extents of their ranges (Figure \ref{fig:range_plot}). Within each plot, a maximum of 10 seedlings were sampled. Seedling mortality creates a narrow bottleneck for tree survival in closed canopy tropical forests, seedlings are particularly sensitive to environment stress. Previous studies having demonstrated seedling vulnerable to climate change \citep{REF}. \todo{something about dispersal not catching up at the same rate}. To minimise the chance of pseudo-replication of sampled seedlings, seedlings closer than 5 m to another seedling were excluded from the analysis, as it cannot be guaranteed that the stems are not connected by a stolon or rhizome, it also ensures that competition radius measurements are truly independent. Within a cluster of seedlings, each seedling was assigned a number and a random number table was used to choose a single seedling for measurement.

Proxies for photosynthetic efficiency \todo{or is it capacity???} were measured on the highest fully-expanded leaf of each seedling. Leaf photosynthetic efficiency can be used as an indicator of physiological stress levels. Plants with a lower photosynthetic efficiency are more stressed than those with a higher efficiency. Chlorophyll-\textit{a} fluorescence was measured using a Walz mini-PAM II (Walz Effeltrich, Germany), on a randomly selected area of adaxial leaf surface, avoiding prominent leaf veins according to \citep{}. Chlorophyll-\textit{a} measurements were used to calculate F\textsubscript{v}/F\textsubscript{m} according to \citet{Genty1989}:

\begin{equation} \label{eq:fvfm}
F_v/F_m = (F_m - F_o)/F_m
\end{equation}

Where $F_m$ is the maximal fluorescence in the dark and $F_o$ is the minimal fluorescence in the dark \citep{Maxwell2000}. Fluorescence measurements were taken after exposing the seedling to 30 minutes of \todo{total} darkness, \todo{to ensure complete dark adaptation} \citep{Campbell2007}. Dark-adapted F\textsubscript{v}/F\textsubscript{m} measures the photosynthetic efficiency of the leaf by relaxing the reaction centres prior to the fluorescence measurement. F\textsubscript{v}/F\textsubscript{m} is preferable to other chlorophyll fluorescence measures as it removes the noise created by environmental conditions at the time of measurement, instead providing a measure of the underlying photosynthetic efficiency. A reduction in F\textsubscript{v}/F\textsubscript{m} is indicative of plant stress. Here, individuals with F\textsubscript{v}/F\textsubscript{m} values <0.7 are said to be experiencing stress \citep{Maxwell2000}. 

In addition to F\textsubscript{v}/F\textsubscript{m} leaf relative chlorophyll content was measured using a multi-spectral SPAD-meter (Minolta SPAD-502Plus, Spectrum Technologies, Plainfield, Illinois, USA). To account for variation in chlorophyll content across the leaf \todo{REF}, SPAD measurements were taken at three random points on the leaf and the geometric mean was used in analyses. 

Leaf and whole-plant morphological measurements 

To assess adult-seedling competition interactions an adapted version of the Iterative Hegyi Index was implemented \citep{Hegyi1974, Lee2004, Seifert2014}. Our adapted `Iterative Seedling Index' (ISI) uses adult tree trunk diameter at \textasciitilde 1.3 m from ground level  (Diameter at Breast Height, DBH) and the distance of trees from the seedling to calculate an index for each seedling, higher values indicate greater competition pressure from surrounding adult trees:

\begin{equation}
\label{eq:ISI}
ISI_i = log(\sum_{j=1}^n (\frac{1}{{DIST_i}_j} D_j))
\end{equation}

where $D_j$ is the DBH of a competitor tree and ${{DIST_i}_j}$ is the euclidean distance between seedling $i$ and competitor tree $j$. ISI was log transformed for analysis, as results spanned multiple orders of magnitude. The `iterative' aspect refers to the selection of competitor trees. The radius around the seedling is divided into 12 30\textdegree  sectors, only the nearest tree >10 cm DBH within each sector is measured (Figure \ref{fig:hegyi}). The size of the competition radius ($C_R$) is defined as:

\begin{equation}
\label{eq:CR}
C_R = 2 \times \sqrt{\frac{10,000}{N}}
\end{equation}

where $N$ is the number of trees \textgreater10 cm DBH per ha (stand density). Stand density data was taken from ABERG census data within each plot (ABERG unpublished data) and used to estimate $C_R$ for VC, for which no stand density data exists (\todo{Supplementary material}). $C_R$ was rounded to the nearest metre for ease of measurement (Table \ref{table:CR}). An iterative selection method for competitive trees assumes that if the path between two trees is blocked, the intensity of competition between them will be greatly reduced \citep{Gadow1999}.


\subsection*{Statistical Analysis}

A matrix of linear mixed effects models were compared to test for the presence and strength of the relationship between each competition variable and each plant trait. All model variables were standardised to allow easy comparison of effect sizes, according to \citep{Gelman2008, Grueber2011, Gelman2014}. Model quality was compared using Akaike Information Criteria (AIC) \citep{Akaike1998}, Akaike weights ($W_i$), and fixed effect pseudo-R\textsuperscript{2} values ($R_M^2$). 

\todo{Add linear mixed model set up as an equation or schematic diagram or something}

Simple linear regressions investigated the relationship between competition intensity and elevation. All statistical analyses were conducted using R, version 3.2.4 \citep{R2016}.

\section*{Results}
   $Chl = 0.53e^{\begin{matrix} 0.0364 \times SPAD \end{matrix}}$

\section*{Discussion}

\section*{Conclusion}


Forest structure based competition affects physiological stress independently of elevation

%--------------------------------------------------------------
\end{document}
%--------------------------------------------------------------
